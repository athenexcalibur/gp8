\documentclass[12pt]{article}
\usepackage[utf8]{inputenc}
\usepackage[T1]{fontenc}
\usepackage{graphicx}
\usepackage{xcolor}
\usepackage{hyperref}
\usepackage{multirow}

\input{_defs}
%custom Requirements titles
\usepackage{titlesec}
%\usepackage{hyperref}
%Funtional
\titleclass{\requirement}{straight}[\subsubsection]
\newcounter{requirement}
\titleformat{\requirement}
  {\color{color1}\sffamily\bfseries}{}{0em}
  {\refstepcounter{requirement}FR \therequirement:~}
\titlespacing*{\requirement}{0pt}{3.25ex plus 1ex minus .2ex}{1.5ex plus .2ex}

\setcounter{tocdepth}{4}
\makeatletter
  \def\toclevel@requirement{5}
  \def\l@requirement{\@dottedtocline{5}{3.8em}{3.2em}}
\makeatother
%Non-Functional
\titleclass{\qrequirement}{straight}[\subsubsection]
\newcounter{qrequirement}
\titleformat{\qrequirement}
  {\color{color1}\sffamily\bfseries}{}{0em}
  {\refstepcounter{qrequirement}QR \theqrequirement:~}
\titlespacing*{\qrequirement}{0pt}{3.25ex plus 1ex minus .2ex}{1.5ex plus .2ex}

\setcounter{tocdepth}{4}
\makeatletter
  \def\toclevel@qrequirement{5}
  \def\l@qrequirement{\@dottedtocline{5}{3.8em}{3.2em}}
\makeatother
%%%


%requirements table enviromnent
\newenvironment{reqtable}
{
    \medskip
    \begin{tabular}{|p{3.5cm}|p{11cm}|}
    \hline
}
{\end{tabular}}


\usepackage{lipsum}

%%%%%%%%%%%%%%%
% Title Page
\title{Cupboard:\\Requirements Specification}
\author{Team 8:\\Clare Doran, Abdul Ghani, Ucizi Mafeni, Luke Needham, Soumya Singh}
\date{\today}
\summary{
A web app built with the intention of helping reduce food waste.
}
%%%%%%%%%%%%%%%

\begin{document}
\maketitle

\tableofcontents
\clearpage


%glossary
%define "active" listing
%define listing
%define immutable
%define trade
\section{Introduction}

7.5 million tonnes of food went to waste in 2015, 4.4 million tonnes of which was edible.
This has a retail value of around £13 billion, and is associated with 19 million tonnes of carbon dioxide emmisions
(which is around a quarter of that emitted by road traffic in the UK)\cite{wrap}.
Meanwhile food banks are becoming busier every year \cite{trussel} and 8.4 million people in the UK alone struggling to pay for food \cite{voth}.\\
Much of this is simply because the expiry date has passed. However, `dates are not an indicator of the product’s safety' \cite{USDA} and food can remain perfectly edible after the expiry date.
From an environmental, economical, and ethical point of view this is a completely needless travesty.\\
Cupboard aims to be a system that allows users to painlessly find and share food
that would otherwise go to waste. A priority is speed and ease of use, as it is
too easy to just throw away food. 
It will also allow users to arrange collection in a manner that does not force
them to disclose their house address or any other details they wish to keep
private.

The main sections of this document are:
\begin{itemize}
	\item \textbf{Project Scope:} Here we give an overview about what our system will (and won't) do. Identification of the target audience and possible future features.
	\item \textbf{Domain analysis:} An analysis of similar products already in place.
	\item \textbf{Proposed Deliverables:} How we plan to undergo the project.
	\item \textbf{Identified, Risks, Assumptions, Dependencies and Constraints:}
        A discussion of things we must consider and keep in mind as we move forward.
	\item \textbf{Solution Requirements:} Planned functional and non-functional requirements.
	\item \textbf{Development Approach:} A brief overview of the technologies we plan to use in order to develop the product.
\end{itemize}


\section{Project Scope}
\subsection{What Cupboard will do}
The aim is to become a go-to everyday service that people will use `by default'
without even thinking about wasting food. This means users will be able to
easily list items with a minimal amount of effort.\\
There will be a simple points-based system and a ranking system in order to
encourage sharing of food and to reassure other users that the quality of food is
likely to be good. When some food has been successfully exchanged both
parties will get points and will be able to rate the other based on their experience. If users wish to
participate, there will be a simple public leaderboard.\\
Users can set their dietary requirements (allergies, religious etc.) and have
results automatically filtered. This will allow users to find food even more quickly, as they do not need to manually filter out innapropriate foods.\\
There will be a messaging system (in order to arrange collection) and a comments
system (in order to view interest in the item and ask questions). \\
We also are strictly not-for-profit, and will not allow users to demand money for their leftovers. There will be no
advertisements on our site.

\subsection{What Cupboard won't do}
We wish to follow the UNIX philosophy of `doing one thing and doing it well'. As mentioned above,
our `one thing' is the prevention of food waste by donating food to people who want it.
We have user accounts, a messaging system, comments, and a leaderboard only to facilitate the sharing of food.
Social media features (such as friend requests, status updates, etc) will not feature in our product.

\subsection{Target audience}
Our plans are to keep the interface incredibly simple and as accessible as possible. 
Ideally, anyone who can browse the internet will be able to find and share food via our platform.\\
That said, we have identified four main user groups which we think will benefit most from our service:

\begin{itemize}
	\item \textbf{Families:} When cooking for a group it can be hard to know in advance how much you should prepare.
			It is common to prepare more than you need and throw away what remains.
			Our service must therefore be family friendly, which means no explicit language, advertisements etc.
	\item \textbf{Stores and resturants:} Stores are reluctant so sell food after the `best-before' date, despite it being perfectly legal \cite{govuk}.
			Resturants commonly throw away completley edible trimmings in order to make meals more aesthetic. Our service will allow them to advertise food,
			similarly to the `FoodCloud' service mentioned below in the Domain Analysis.
	\item \textbf{Students:} Preparing single meals can be tough and wasteful. Our service will allow students to find dinner if they can't cook, or share food
			with other students if they do cook and find they've prepared too much.
	\item \textbf{The needy:} Most importantly, our service will allow those who do not have the means to afford food to feed themselves whilst preventing food waste.
\end{itemize}

\subsection{Future features}
The features we plan to implement are plenty and should make a sizeable dent in our food wastage problem. There are, however, several features
that we would implement after the initial launch of the system:
\begin{itemize}
	\item \textbf{Featured listings:} Large stores and resturants may sign up to be `featured users'. Generally, food from these places will be of high-quality
		(i.e, still boxed, only just past the sell-by date) and users will feel more comfortable picking up food from a store. Listings from these users will
		be promoted in search results. \textit{Note: this is not a paid feature}.
	\item \textbf{Mobile application:} To make the service even more accessible we may develop a mobile application. This will include a barcode scanner, so that users
		can list something in a matter of seconds.
	\item \textbf{Statistics:} Anonymous statistics can be gathered so that we may find out roughly how much food waste our system is preventing.
	\item \textbf{Volunteer service:} Particularly vunerable people (such as the elderly or disabled) may not be able to leave the house easily. There are charities
		that help by delivering food to these people. One such charity is mentioned in the next section.
	\item \textbf{Donations:} As mentioned above, we will not be for profit. However, we might collect donations to cover server costs, with the rest being distributed
		to relevant charities.
	\item \textbf{Education:} It may be useful to develop a short series of guides, containing tips and tricks to help people make better use of their food.
\end{itemize}


\section{Domain Analysis}

The domain of our project encompasses online marketplaces and sites for purchasing food.
There are a number of sites which already occupy these domains,
the design of which we will draw from in designing our own site.

\begin{itemize}

    \item eBay \cite{ebay}: An online marketplace through which users can both list and
        purchase almost anything. When listing an item, users must provide at
        least one photo and information such as title, condition and type, as
        well as an optional description and additional photos.

        This is important as it forces users to provide enough information
        for buyers, and discourages fake listings.
        Users search for items by name, and can sort results in a number of
        ways, which is very useful, however it lacks more options for filtering
        results.
        
        Perhaps the best feature of the site is that every listing displays
        the seller’s username, along with a score and the percentage of all
        the feedback the user has received which is positive.
        This allows buyers to gauge how trustworthy the seller is, and is a
        feature which we feel will work well in our system also.


    \item foodsharing.de \cite{foodsharing}: A food sharing sharing site, currently only available
        in Germany.
        It reduces food waste by allowing users to list food they wish to give
        away, and then arrange a time and place for the volunteer
        “Lebensmittelretter Innen” to collect the food and deliver it to
        whoever wanted it.
        This method removes the risk of potentially dangerous or illegal
        substances being exchanged, as all items are checked by the trusted
        volunteers.
        
        However it introduces the problem of having to have a large number of
        volunteers willing to give up their time. The site itself is quite well
        designed, but navigation is sometimes difficult.

        Every item of food listed has its own page, but we found these pages to
        be lacking, listing only a very brief description most of the time.
        The listings are also all anonymous and there is no system of rating
        users, as in eBay for example, which would allow users to check how
        trustworthy the user providing the food is.
    
    \item Olio \cite{olio}: A mobile app with a strong UK presence.
        Users can share food by manually collecting/delivering it, or through
        "Drop Boxes" - local stores or cafés where users can drop off food
        they'd like to be shared.
        
        However the app has received many negative reviews from users frustrated
        that there is no food they can get near them, which is an issue our
        system is also likely to face, particularly soon after launch.
        
    \item FareShare \cite{fareshare}:
		FareShare is a London-based charity that redistributes food from restaurants
		and grocery stores to other local charities and community-groups. They are similar
		to our proposed platform, and even have a `FoodCloud' service in which large stores
		can alert local charities that they have food available for colllection.\\
		One issue with FareShare is that not anyone can quickly share or find food. There is no available listing of locally available leftovers.
		Local resturants and stores can sign up to donate food, and local charities can sign up to recieve food, however
		if (say) a family of 4 want to donate some leftovers they cannot do so with this service.
\end{itemize}

An interesting feature of the 2 existing food sharing apps listed above is that
their search functions and catalogs are strongly based on location.
They place a lot of emphasis on an interactive map to find items, contrary to
eBay’s method of having the user search for items by name, and then allowing
the user to filter results, indicating perhaps that users are more concerned
about how far they have to travel for food, rather than what that food is.

We feel our system should combine both of these functions, with a search bar
and filtering methods like eBay, but with results sorted by distance from the user.

\pagebreak

\section{Proposed Deliverables}
\subsection{Division of Tasks}

Though all members shall inevitably contribute to both sides of the task,
the major components of the project have been split as follows:

\begin{table}[ht!]
    \centering
    \begin{tabular}{|l|l|l|}
        \hline
        \textbf{Task}                 & \textbf{Subtask}                         & \textbf{Team Member} \\
        \hline
        \multirow{4}{3.5cm}{Frontend} & Website Layout Wireframe                 & Luke                 \\
        \cline{2-3}
                                      & Logo and Branding Design                 & Clare                \\
        \cline{2-3}
                                      & HTML/CSS/Bootstrap                       & Ucizi                \\
        \cline{2-3}
                                      & Dynamic Features (Javascript and JQuery) & Ucizi                \\
        \hline
        \multirow{7}{3.5cm}{Backend}  & User Account --- Registration and Login  & Abdul                \\
        \cline{2-3}
                                      & Search                                   & Soumya               \\
        \cline{2-3}
                                      & Listings/Posts                           & Soumya               \\
        \cline{2-3}
                                      & Messaging                                & Abdul                \\
        \cline{2-3}
                                      & Comments                                 & Soumya               \\
        \cline{2-3}
                                      & User Ratings and Scores                  & Abdul                \\
        \cline{2-3}
                                      & Database Design and Implementation       & Abdul \& Soumya      \\
        \hline
    \end{tabular}
    \label{tab:tasks}
\end{table}
\pagebreak
\subsection{Milestones and Deadlines}

\begin{figure}[ht!]
    \centering
    \includegraphics[width=0.7\textwidth]{images/plan.png}
    \caption{Project Gantt Chart}
    \label{fig:gantt-chart}
\end{figure}

\pagebreak
\begin{table}[ht!]
    \centering
    \begin{tabular}{|l|l|}
        \hline
        Date & Task\\
        \hline
        \multirow{2}{2cm}{30/10/16} & Database Design\\
        & Initial UI Mockups\\
        \hline
        \multirow{2}{2cm}{15/11/16} & User Account Management\\
        & Website Template\\
        \hline
        \multirow{4}{2cm}{16/12/16} & Search Functions\\
        & Final Mockups\\
        & Full Database Implementation\\
        & Main and auxiliary pages\\
        \hline
        \multirow{2}{2cm}{31/01/17} & Post Submission\\
        & Search Page\\
        \hline
        \multirow{2}{2cm}{24/02/17} & Messaging\\
        & Listing Pages\\
        \hline
        \multirow{2}{2cm}{16/03/17} & Comments\\
        & User Pages\\
        \hline
        \multirow{2}{2cm}{31/03/17} & User Ratings\\
        & JS Effects\\
        \hline

    \end{tabular}
    \caption{Hard Deadlines}
    \label{tab:deadlines}
\end{table}
\pagebreak
\section{Identified Risks, Assumptions, Dependencies and Constraints}
\subsection{Risks}
\begin{itemize}

    \item It is possible that users may exploit the rating system,
        through the use of fake reviews and ratings,
        to boost their score and rating, and therefore trick
        other users into believing they are trustworthy when this is not the case.

    \item Users may exploit the system to list and distribute non-food items,
        and particularly nefarious users may even use it to distribute illegal substances.
    
    \item Users may (knowingly or otherwise) distribute food which is dangerous
        to consume, such as if it has gone off.
    
    \item Users may lure others, under the guise of giving them food, into
        potentially perilous predicaments.
    
    \item If the system’s sole database goes down the whole site will be
        useless, since all the data is stored in a single place.
    
    \item It is possible that the system, particularly its databases, could be hacked.
        This could lead to user’s private data, particularly sensitive
        information such as address and email address, being leaked and abused.
    
    \end{itemize}

Should any of these risks occur, or should anything else unfortunate occur
to anyone, but particularly minors, through use of the system,
we may be held accountable.
This in itself is also a risk.
To mitigate this, we will ensure that we mitigate all other risks as much
as possible, and force users to sign agreements waiving all responsibility from us.
The ways we mitigate these risks are detailed in Functional Requirements.

\subsection{Assumptions \& Dependencies}
\begin{itemize}

    \item We assume Google maps will be able to find the vast majority of
        addresses entered by the user.
        This location functionality is thus completely dependent on the Google Maps API.

    \item We assume that when a user searches for something they will find a
        decent number of results.
        This is dependent on enough users listing items on the system,
        otherwise, the system will be useless to people wanting to receive food.

\end{itemize}

\subsection{Constraints}
\begin{itemize}

\item We need to ensure that no user’s private information, including address
    and email address, is unnecessarily stored or made public, in order to
    avoid putting their security at risk.
    %In particular, we need ensure the privacy of children under 18.

\item Due to the risk involved in regarding strangers meeting to collect items,
    we will restrict our app to users who are 18 years and older

\item In order to allow it to run quickly even on devices with poor internet,
    the system will be limited in the
    number of performance-detrimental luxuries it can have.

\item Due to a lack of time and relative lack of experience,
    the system will likely not be as attractive or efficient as it would be ideally.

\end{itemize}

\section{Solution Requirements}
\subsection{Functional Requirements}
%move section of speech to development approach
N.B. Due to the highly time-sensitive nature of the project, any low priority
requirements (and thier linked dependencies) may not be implemented.
As such, all low priority requirements are inherently unessential to the
system, and it will remain fully operational with or without them.

\requirement{"About Us" page}
\label{fr:about-us}

\begin{reqtable}
    Description        & All users should have access to a page that contains 
                        information about the web app. This includes:
                        
                        \begin{itemize}
                            \itemsep-1em
                            \item A short description about the purpose of the
                                app
                            \item Contact information
                            \item Terms and Conditions of use
                            \end{itemize}
                        \\
    \hline
    Priority           & Medium\\
    \hline
    Dependencies       & None\\
    \hline
    Expected results   & User able to access the page.
                        
                        Page contains all the information stated in the
                        description.\\
    \hline
    Exception Handling & 
    			\begin{description}
                            \itemsep0em
                            \item [Page not found:] refresh page, check URL
			    \item [Cannot find after refresh:] report error
                                to webmaster
                        \end{description}
                        \\
    \hline
\end{reqtable}


\requirement{Error-Reporting System}
\label{fr:error-reporting}

\begin{reqtable}
    Description        & Users should be able to report any errors they notice
                        to the webmaster.
                        Each error report should have:

                        \begin{itemize}
                            \itemsep-1em
                            \item A title, which is a brief description of the
                                error (max. 50 chars)
                            \item A more detailed description of the error
                                (max. 500 chars)
                        \end{itemize}
                        \\
    \hline
    Priority           & Medium\\
    \hline
    Dependencies       & \autoref{fr:about-us}\\
    \hline
    Expected results   & User can successfully send error report\\
    \hline
    Exception Handling & 
                        \begin{description}
                            \itemsep0em
                            \item [User cannot send error report:] use email on
                                "About us" page to contact webmaster
                        \end{description}
                        \\
    \hline
\end{reqtable}


\requirement{User Feedback System}
\label{fr:user-feedback}

\begin{reqtable}
    Description        & Users should be able to provide feedback regarding
                        their experience using the web app to the webmaster.
                        This should be done in the form of a feedback form,
                        which should have the following fields:

                        \begin{itemize}
                            \itemsep-1em
                            \item A title, which is a brief description of the
                                feedback report(max. 50 chars)
                            \item The main feedback report (max. 1000 chars)
                        \end{itemize}
                        \\


    \hline
    Priority           & Low\\
    \hline
    Dependencies       & \autoref{fr:error-reporting} (Error Reporting)\\
    \hline
    Expected results   & User's can successfully send feedback report\\
    \hline
    Exception Handling & 
                        
                        \begin{description}
                            \itemsep0em
                            \item [User unable to send feedback form]: report 
                                error to webmaster
                        \end{description}
                        \\
    \hline
\end{reqtable}


\requirement{User sign up}
\label{fr:user-sign-up}

\begin{reqtable}
    Description        & The User should be able to create an account by 
                        following the registration process.

                        Registration will involve:

                        \begin{itemize}
                            \itemsep-1em
                            \item Getting the user registration details
                                (see \autoref{fr:registration-details}).
                            \item Getting the user to accept that they are at 
                                least 18 years old.
                            \item Getting the user to agree to the Terms and
                                Conditions of use (see \autoref{fr:about-us}).
                        \end{itemize}
                        \\
    \hline
    Priority           & High\\
    \hline
    Dependencies       & \autoref{fr:error-reporting},
    \autoref{fr:registration-details} (Get user registration details),
    \autoref{fr:about-us} (About us page)
                        \\
    \hline
    Expected results   & Following account creation, the user will be able to 
                        login using their provided username and password\\
    \hline
    Exception Handling & 
                        
                        \begin{description}
                            \itemsep0em
                            \item [User unable to create account:] report error
                                to webmaster
                        \end{description}
                        \\
    \hline
\end{reqtable}

\requirement{Get User Registration Details}
\label{fr:registration-details}

\begin{reqtable}
    Description        & During sign-up, the user should provide the following
                        details in order to successfully complete the sign-up
                        process:

                        \begin{itemize}
                            \itemsep-1em
                            \item username (must be unique)
                            \item email address
                            \item password
                            \item physical address
                            \item post code (must be valid)
                        \end{itemize}
                        
                        In addition, the user should also be given the option
                        to fill in their Dietary Requirements and Allergy 
                        Information. However, it is not mandatory to fill in 
                        these details in order to complete registration.

                        \\
    \hline
    Priority           & High\\
    \hline
    Dependencies       & \autoref{fr:error-reporting},
    \autoref{fr:validation} (Validate User Details), 
    \autoref{fr:dietary-requirements} (Dietary Requirements), 
    \autoref{fr:allergy-information} (Allergy Information)\\
    \hline
    Expected results   & User inputs all valid fields and proceeds to complete
                        registration\\
    \hline
    Exception Handling & 
                        
                        \begin{description}
                            \itemsep0em
                            \item [User unable to fill in registration details:]
                                report error to webmaster
                            \item [User inputs invalid details:] should be
                                validated as described in \autoref{fr:validation}
                        \end{description}
                        \\
    \hline
\end{reqtable}

\requirement{Validate Registration Details}
\label{fr:validation}

\begin{reqtable}
    Description        & Information in the registration form should be
                        validated as follows:
                        
                        username: checked against database to see if its unique

                        email-address: adheres to general email structure 
                        (<someuser>@<somehost>.<com/net\ldots>)

                        password: minimum length of 7 characters, alphanumeric,
                        contains one symbol from valid symbol list
                        (refer to glossary)

                        postcode: is actual valid postcode 
                        (check in UK postcode directory)

                        If there exists an invalid input on submission,
                        the form must be rejected and the user must be
                        notified of their error
                        (i.e highlight incorrect field and place error message
                        next to it)
                        \\
    \hline
    Priority           & High\\
    \hline
    Dependencies       & \autoref{fr:error-reporting}\\
    \hline
    Expected results   & Validation system correctly highlights any invalid
                        fields\\
    \hline
    Exception Handling & 
                        
                        \begin{description}
                            \itemsep0em
                            \item [Validation system doesn't meet specified 
                                requirements:]
                                report error to webmaster
                        \end{description}
                        \\
    \hline
\end{reqtable}

\requirement{Account Activation}
\label{fr:activation}

\begin{reqtable}
    Description        & 
                        Once the user has provided the registration details,
                        an email with an account activation link
                        should be sent to the provided email address. The user
                        should then be able to follow the activation link in
                        order to activate their account.\\
    \hline
    Priority           & Low \\
    \hline
    Dependencies       & \autoref{fr:error-reporting}\\
    \hline
    Expected results   & After the user follows the activation link,
                        they should be able to login to their account using
                        the login details they provided during registration\\
    \hline
    Exception Handling & 
                        \begin{description}
                            \itemsep0em
                            \item [Incorrect email address provided:]
                                the user should have the
                                option to correct the provided email address.
                            \item [Activation link doesn't work:] the user
                                should have the option to request another
                                activation link.
                            \item [Activation system doesn't work:] report
                                error to webmaster
                        \end{description}
                        \\
    \hline
\end{reqtable}

\requirement{Password Reset Facility}
\label{fr:password-reset}

\begin{reqtable}
    Description        & If a registered user forgets their password, they
                        should be able to request a password reset link.
                        
                        They should provide a valid email address within
                        the request. If the provided email address is linked
                        to an account on the system, a password reset link
                        should be sent to that email.

                        Upon following the password reset link, the user should
                        be able to set a new password.
                        \\
    \hline
    Priority           & Low\\
    \hline
    Dependencies       & \autoref{fr:validation}\\
    \hline
    Expected results   & User able to successfully reset password\\
    \hline
    Exception Handling & 
                        
                        \begin{description}
                            \itemsep0em
                            \item [User unable to reset password:] report error
                                to webmaster.
                            \item [Email doesn't meet validation
                                standards set in \autoref{fr:validation}:]
                                Prevent submission of password reset request.
                                Notify user that invalid email was entered.
                        \end{description}
                        \\
    \hline
\end{reqtable}


\requirement{User Dashboard}
\label{fr:user-dashboard}

\begin{reqtable}
    Description        & Following login, all registered users should have
                        access to a personal
                        dashboard where they can do the following:
                        
                        \begin{itemize}
                            \itemsep-1em
                            \item edit their personal details
                            \item edit their current dietary requirements
                            \item edit their allergy information
                            \item view their user score and rating
                        \end{itemize}

                        \\
    \hline
    Priority           & High\\
    \hline
    Dependencies       & \autoref{fr:edit-profile} (Edit Profile),
    \autoref{fr:dietary-requirements},
    \autoref{fr:allergy-information},
    \autoref{fr:user-score} (User Score),
    \autoref{fr:user-ratings} (User Rating)
    \\
    \hline
    Expected results   & User is able to access and edit all personal
                        information from the dashboard\\
    \hline
    Exception Handling & 
                        \begin{description}
                            \itemsep0em
                            \item [Registered and logged-in user can't access
                                dashboard:] notify webmaster of error
                        \end{description}
                        \\
    \hline
\end{reqtable}


\requirement{Edit Profile}
\label{fr:edit-profile}

\begin{reqtable}
    Description        & A user should be able to change their personal details.
                        This includes:

                        \begin{itemize}
                            \itemsep-1em
                            \item their username
                            \item their email address
                            \item their password
                            \item their dietary requirements
                            \item their allergy information
                        \end{itemize}

                        All changes to personal details will be validated under
                        the standards mentioned in
                        \autoref{fr:validation}.

                        With regards to dietary requirements and allergy
                        information, users will be able to select whichever
                        categories listed in \autoref{fr:dietary-requirements}
                        and \autoref{fr:allergy-information} that personally
                        apply to them.
                        \\
    \hline
    Priority           & High\\
    \hline
    Dependencies       & \autoref{fr:validation},
                        \autoref{fr:dietary-requirements},
                        \autoref{fr:allergy-information}
                        \\
    \hline
    Expected results   & User able to successfully change personal details.\\
    \hline
    Exception Handling & 
                        \begin{description}
                            \itemsep0em
                            \item [User unable to save changes:]
                                Report error to webmaster
                            \item [New details don't satisfy validation standards:]
                                Invalid fields must be highlighted with hint
                                (as to why field is Invalid) shown
                        \end{description}
                        \\
    \hline
\end{reqtable}


\requirement{Dietary Requirements}
\label{fr:dietary-requirements}

\begin{reqtable}
    Description        & Dietary requirements should be split into
                        the following categories:

                        \begin{itemize}
                            \itemsep-1em
                            \item Halal
                            \item Kosher
                            \item Vegetarian
                            \item Other
                        \end{itemize}
                        \\
    \hline
    Priority           & High\\
    \hline
    Dependencies       & \autoref{fr:error-reporting}\\
    \hline
    Expected results   & Dietary requirements meet specified requirements.\\
    \hline
    Exception Handling & 
                        \begin{description}
                            \itemsep0em
                            \item [Dietary requirements doesn't meet specification:]
                                report error to webmaster.
                        \end{description}
                        \\
    \hline
\end{reqtable}


\requirement{Allergy Information}
\label{fr:allergy-information}

\begin{reqtable}
    Description        & Allergies should be split into
                        the following categories:

                        \begin{itemize}
                            \itemsep-1em
                            \item Nuts
                            \item Gluten
                            \item Soy
                            \item Other
                        \end{itemize}\\
    \hline
    Priority           & High\\
    \hline
    Dependencies       & \autoref{fr:error-reporting}\\
    \hline
    Expected results   & Allergy information meets specified requirements.\\
    \hline
    Exception Handling & 
                        \begin{description}
                            \itemsep0em
                            \item [Allergy information doesn't meet specification:]
                                report error to webmaster.
                        \end{description}
                        \\
    \hline
\end{reqtable}


\requirement{User Score}
\label{fr:user-score}

\begin{reqtable}
    Description        & All Users will have a personal score, starting from 0,
                        which will increase as the user gains ‘points’. Points
                        will be earned for every trade, with 10 points being
                        given to the provider, and 1 point being given to the
                        receiver.
                        \\
    \hline
    Priority           & Low\\
    \hline
    Dependencies       & \autoref{fr:error-reporting}\\
    \hline
    Expected results   & User is awarded points as specified above.\\
    \hline
    Exception Handling & 
                        \begin{description}
                            \itemsep0em
                            \item [User notices error in current score:]
                                Report error to webmaster
                        \end{description}
                        \\
    \hline
\end{reqtable}

\requirement{Score Leaderboard}

\begin{reqtable}
    Description        & The User Score Leaderboard should display the scores
                        of all users in descending order
                        (i.e highest score first).
                        
                        The Leaderboard should be visible to all users.

                        When a registered user views the Leaderboard, their
                        position on the Leaderboard should be clearly highlighted
                        so its easy to find.

                        There should also be a shortcut that allows that 
                        allows the user to instantly view their position on 
                        the Leaderboard.
                        \\
    \hline
    Priority           & Low\\
    \hline
    Dependencies       & \autoref{fr:error-reporting},
    \autoref{fr:user-score}\\
    \hline
    Expected results   & Leaderboard displays all user scores.
    
                        All user scores displayed in descending order\\
    \hline
    Exception Handling & 
                        \begin{description}
                            \itemsep0em
                            \item [Multiple users are tied with the same points:]
                                They all get the same position number, and the
                                next lowest ranked user is given a position equal
                                to the number of users with a higher score,
                                plus one.
                                e.g. if there are 4 users tied at 3rd, the next
                                user will 8th.
                        \end{description}
                        \\
    \hline
\end{reqtable}



\requirement{User Ratings}
\label{fr:user-ratings}

\begin{reqtable}
    Description        & All Users will have a personal rating from 0 to 5,
                        calculated as the average of all the ratings received
                        from other users. Whenever a user either provides or
                        collects an item, the other user involved in the trade
                        will rate them out of 5, and this will change their
                        average rating accordingly.

                        The publisher of the listing will be able to rate the
                        collector and vice versa once the item is “inactive”. 
                        For the publisher this should be done from the listing
                        history page (see \autoref{fr:user-listing-page}).
                        For the collector, this should be done from the order
                        history page (see \autoref{fr:orders})\\
    \hline
    Priority           & Low\\
    \hline
    Dependencies       & \autoref{fr:error-reporting},
    \autoref{fr:user-listing-page} (User Listing Page),
    \autoref{fr:orders} (Orders Page)\\
    \hline
    Expected results   & User has personal rating and it is correct.
                        
                        User able to rate other user involved in any trade.\\
    \hline
    Exception Handling & 
                        \begin{description}
                            \itemsep0em
                            \item [User unable to give rating:]
                                Report error to webmaster
                            \item [User notices irregularity in current rating:]
                                Report possible error to webmaster
                        \end{description}
                        \\
    \hline
\end{reqtable}


\requirement{Listing}
\label{fr:listing}

\begin{reqtable}
    Description        &
                        A listing is the core element of the web catalogue.
                        All listings should contain the following attributes:
                        \begin{itemize}
                            \itemsep-1em
                            \item The name of seller, along with their score
                                and rating (immutable).
                            \item A main title describing the item (max. 50 chars)
                            \item A longer description of the item
                                (optional; max. 200 chars).
                            \item The date the listing was added (immutable)
                            \item An expiry date for the food item.
                            \item Checkboxes for which dietary groups this food is not appropriate for.
                            \item The postcode from which the item can be collected
                            \item A listing status (i.e. Active/Inactive)
                            \item 1 primary photo
                                (Will be displayed with listing in
                                public catalogue [\autoref{fr:item-catalogue}])
                            \item Up to 3 secondary photos
                                (Will displayed within listing page [\autoref{fr:listings-page}])
                        \end{itemize}
                        \\
    \hline
    Priority           & High\\
    \hline
    Dependencies       & \autoref{fr:error-reporting}, 
    \autoref{fr:item-catalogue} (Item Catalogue),
    \autoref{fr:listing-status} (Listing Status),
    \autoref{fr:active-listing}(Active Listing),
    \autoref{fr:inactive-listing} (Inactive Listing),
    \autoref{fr:listings-page}\\
    \hline
    Expected results   & All listing items conform to the structure mentioned
                        in the description.\\
    \hline
    Exception Handling & \\
    \hline
\end{reqtable}


\requirement{Listing Status}
\label{fr:listing-status}

\begin{reqtable}
    Description        &
                        At any particular moment in time, all listings should
                        have one (and only one) of the following statuses:

                        \begin{description}
                            \itemsep0em
                            \item [Active:] The listing item is available for
                                collection. (see \autoref{fr:active-listing}
                                For more details)
                            \item [Inactive:] The listing item is no longer
                                available for collection
                                (see \autoref{fr:inactive-listing} for more details.)
                        \end{description}
                        \\
    \hline
    Priority           & High\\
    \hline
    Dependencies       & \autoref{fr:error-reporting},
    \autoref{fr:active-listing},
    \autoref{fr:inactive-listing}\\
    \hline
    Expected results   & All listings are either active or inactive\\
    \hline
    Exception Handling & 
                        \begin{description}
                            \itemsep0em
                            \item [Listings don't meet status specification:]
                                report error to webmaster
                        \end{description}
                        \\
    \hline
\end{reqtable}


\requirement{Active Listing}
\label{fr:active-listing}

\begin{reqtable}
    Description        & All active listings are visible on the public catalogue.

                        Any registered user (except the publisher) should be
                        able to make an offer to commit to collecting a listed
                        item. Any item should have a maximum of one offer.

                        If another user has already offered to receive an item,
                        it will remain active, but will become unavailable, so
                        no other user will be able to place another offer.
                        \\
    \hline
    Priority           & High\\
    \hline
    Dependencies       & \autoref{fr:error-reporting}, 
    \autoref{fr:item-catalogue},
    \autoref{fr:commit} (Commit to collect)\\
    \hline
    Expected results   & All active listings meet the specifications in the
                        description above.\\
    \hline
    Exception Handling & 
                        \begin{description}
                            \itemsep0em
                            \item [Active listing doesn't meet specification standards:]
                                report error to webmaster.
                        \end{description}
                        \\
    \hline
\end{reqtable}


\requirement{Inactive Listing}
\label{fr:inactive-listing}

\begin{reqtable}
    Description        & 
                        Inactive listings are not visible on the public
                        catalogue. A listing item can only be made inactive by
                        a by the publisher manually setting it to inactive.

                        Once an listing item is made inactive, It will no
                        longer be visible on the public catalogue.

                        It will also remain permanently inactive
                        (i.e. the user cannot reset it to active).
                        \\
    \hline
    Priority           & High\\
    \hline
    Dependencies       & \autoref{fr:error-reporting},
    \autoref{fr:item-catalogue}\\
    \hline
    Expected results   & All inactive listings meet the specifications in the
                        description above.\\
    \hline
    Exception Handling & 
                        \begin{description}
                            \itemsep0em
                            \item [Active Listing doesn't meet specification standards:]
                                report error to webmaster.
                        \end{description}
                        \\
    \hline
\end{reqtable}


\requirement{Listings Page}
\label{fr:listings-page}

\begin{reqtable}
    Description        & All listings will have a specific page, which will
                        contain all the listing information (see \autoref{fr:listing}).
                        When on this page users shall have the option to commit
                        to collecting the item, or to add it to their watchlist.
                        \\
    \hline
    Priority           & High\\
    \hline
    Dependencies       & \autoref{fr:error-reporting},
    \autoref{fr:listing}\\
    \hline
    Expected results   & All listings have a listing page.
                        All listing information present on page as described above.\\
    \hline
    Exception Handling & 
                        \begin{description}
                            \itemsep0em
                            \item [Listing page doesn't meet specification standards:]
                                report error to webmaster.
                        \end{description}
                        \\
    \hline
\end{reqtable}


\requirement{Comments Board}
\label{fr:comments}

\begin{reqtable}
    Description        & All listing pages should have a comments board,
                        where any registered user can post a comment related to
                        the listing.
                        
                        All comments will be visibile to all users.
                        
                        All comments will be sorted by time of posting
                        (ascending order i.e. most recent comment first).\\
    \hline
    Priority           & Low\\
    \hline
    Dependencies       & \autoref{fr:error-reporting},
    \autoref{fr:listings-page}\\
    \hline
    Expected results   & Registered users able to comment on listing page.
                        
                        All comments correctly sorted by time.\\
    \hline
    Exception Handling & \\
    \hline
\end{reqtable}


\requirement{Item Catalogue}
\label{fr:item-catalogue}

\begin{reqtable}
    Description        & The item catalogue is an index of all active listings.
                        Any user should be able to explore the item catalogue.
                        They should also be able to to apply all of the search
                        filters and sorting options mentioned in 
                        \autoref{fr:filters} and \autoref{fr:sorting}.\\
    \hline
    Priority           & High\\
    \hline
    Dependencies       & \autoref{fr:error-reporting}
    \autoref{fr:filters} (Search Filters),
    \autoref{fr:sorting} (Sorting Search Results)
    \\
    \hline
    Expected results   & Item Catalogue available to all users.
    
                        Users able to implement filters and sorting as
                        described.\\
    \hline
    Exception Handling & \\
    \hline
\end{reqtable}


\requirement{Search}
\label{fr:search}

\begin{reqtable}
    Description        & Users will be able to search for items in the complete
                        catalogue of items, based on the item’s name.

                        Sorting and filtering of search results will be done as
                        described in
                        \autoref{fr:filters} and \autoref{fr:sorting}.

                       After performing a search, the user will be shown a page
                       containing all the active item listings matching the
                       search. Each item displayed will show a preview of the
                       full listing, with the item’s primary photo, name and
                       description, as well as the distance the item is from
                       the user. Upon selecting a result, the user will be
                       taken to the listing’s specific page.\\
    \hline
    Priority           & High\\
    \hline
    Dependencies       & \autoref{fr:error-reporting}
    \autoref{fr:filters},
    \autoref{fr:sorting},
    \\
    \hline
    Expected results   & User can perform a search.

                        All filters and sorting options function as specified.
                        
                        All results displayed as specified.\\
    \hline
    Exception Handling & 
                        \begin{description}
                            \itemsep0em
                            \item [User provides empty search query:] return 
                                entire item catalogue.
                            \item [Search query returns no results:] notify 
                                user that no results were found.
                        \end{description}
    \\
    \hline
\end{reqtable}


\requirement{Sorting Search Results}
\label{fr:sorting}

\begin{reqtable}
    Description        & 
                        By default, all search results will be sorted by
                        distance from the source (location of the listing),
                        to the destination (location of the user performing the
                        search) of the listed item where:
                        \begin{itemize}
                            \itemsep-1em
                            \item The source location is always the collection
                                postcode stated on the item's listing page.
                            \item The destination is, by default, the postcode
                                provided during registration.
                        \end{itemize}
                        
                        The user will be given option to change the destination
                        to the current GPS location of their device.

                        Alternatively, users can sort search results by expiry
                        date. This can be in either ascending  order
                        (i.e. soonest date first), or descending
                        order (i.e. furthest date first).
                        \\
    \hline
    Priority           & High\\
    \hline
    Dependencies       & \autoref{fr:error-reporting},
                        \autoref{fr:item-catalogue}\\
    \hline
    Expected results   & 
                        User can sort results by criteria mentioned above.

                        All results are correctly sorted.
                        \\
    \hline
    Exception Handling & 
                        \begin{description}
                            \itemsep0em
                            \item [User unable to sort results:] Report error to webmaster.
                            \item [Results not sorted as expected:] Report error to webmaste.r
                        \end{description}
                        \\
    \hline
\end{reqtable}


\requirement{Search Filters}
\label{fr:filters}

\begin{reqtable}
    Description        & 
                        Users should be able to filter results by:
                        \begin{itemize}
                            \itemsep-1em
                            \item Dietary Requirements
                            \item Allergens
                        \end{itemize}
                        
                        Additionally the user should have the option to filter
                        out “active” but committed items
                        \\
    \hline
    Priority           & High\\
    \hline
    Dependencies       & \autoref{fr:error-reporting},
    \autoref{fr:dietary-requirements},
    \autoref{fr:allergy-information}
    \autoref{fr:item-catalogue}\\
    \hline
    Expected results   & User able to filter results by criteria mentioned above\\
    \hline
    Exception Handling & 
                        \begin{description}
                            \itemsep0em
                            \item [User unable to filter results:] Report error
                                to webmaster.
                        \end{description}
                        \\
    \hline
\end{reqtable}


\requirement{User’s Past \& Current Listings Page}
\label{fr:user-listing-page}

\begin{reqtable}
    Description        & Any logged-in user should have access to a personal 
                        listings page, from where they can do the following:
                        
                        \begin{itemize}
                            \itemsep-1em
                            \item view their current listings (all active listings)
                            \item view their listing history (all inactive listings)
                            \item add new listings
                            \item edit current listings
                        \end{itemize}
                        \\
    \hline
    Priority           & High\\
    \hline
    Dependencies       & \autoref{fr:error-reporting},
    \autoref{fr:listing},
    \autoref{fr:add-listing} (Add Listing),
    \autoref{fr:edit-listing} (Edit Listing)\\
    \hline
    Expected results   & User can perform all actions listed above.\\
    \hline
    Exception Handling & 
                        \begin{description}
                            \itemsep0em
                            \item [User unable to perform any of the actions:]
                                report error to webmaster.
                        \end{description}
                        \\
    \hline
\end{reqtable}


\requirement{Add Listing}
\label{fr:add-listing}

\begin{reqtable}
    Description        & Any logged in user should be able to add a new
                        listing to the item catalogue. For a listing to
                        be successfully added to the catalogue, It must conform
                        to the specification of a valid listing item
                        (see \autoref{fr:listing} ).

                        All newly added listings will be given an “active” status.\\
    \hline
    Priority           & High\\
    \hline
    Dependencies       & \autoref{fr:error-reporting},
    \autoref{fr:listing}
    \autoref{fr:item-catalogue}\\
    \hline
    Expected results   & User able to add listing.
    
                        Listing has “active” status \\
    \hline
    Exception Handling & 
                        \begin{description}
                            \itemsep0em
                            \item [User unable to add listing:] Report error to webmaster.
                        \end{description}
                        \\
    \hline
\end{reqtable}


\requirement{Edit Listing}
\label{fr:edit-listing}

\begin{reqtable}
    Description        & 
                        All registered users should be able to edit any of their
                        active listings.
                        Users should be able to modify the following attributes:

                        \begin{itemize}
                            \itemsep-1em
                            \item Listing Title
                            \item Listing Description
                            \item Expiry Date
                            \item Dietary Requirements and Allergy Information
                            \item Collection Postcode
                            \item Listing photo(s)
                        \end{itemize}

                        Additionally the user should be able to set the listing
                        to “inactive”. As described by
                        \autoref{fr:inactive-listing}, once
                        set to “inactive”, the user cannot revert the status to
                        “active”, and will be unable to perform any of the
                        modifications listed above.\\
    \hline
    Priority           & High\\
    \hline
    Dependencies       & \autoref{fr:error-reporting},
    \autoref{fr:listing}
    \\
    \hline
    Expected results   & User able to edit listing.

                        Changes successfully saved after edit complete.\\
    \hline
    Exception Handling & 
                        \begin{description}
                            \itemsep0em
                            \item [Edited field(s) don't meet specifiation standards:]
                                prevent the edit, notify user of invalid fields.
                        \end{description}
                        \\
    \hline
\end{reqtable}


\requirement{Commit to collect}
\label{fr:commit}

\begin{reqtable}
    Description        & 
                        A user should be able commit to collecting an item that
                        has an “active” status (given that no other has
                        committed to that item).
                        
                        Once that user has committed
                        to collecting the item it will be added to their orders
                        page (see \autoref{fr:orders}), and a line of communication
                        should be opened between the user and publisher via the
                        messaging facility (see \autoref{fr:messaging})
                        \\
    \hline
    Priority           & High\\
    \hline
    Dependencies       & \autoref{fr:error-reporting},
    \autoref{fr:listing},
    \autoref{fr:messaging} (Messaging),
    \autoref{fr:orders}\\
    \hline
    Expected results   & User able to commit to active item.

                        User cannot commit inactive item.\\
    \hline
    Exception Handling & 
                        \begin{description}
                            \itemsep0em
                            \item [Another user has already committed to the item:]
                                 Any users who subsequently view the shouldn’t
                                 should not have the option to commit to the
                                 same item.
                        \end{description}
                        \\
    \hline
\end{reqtable}


\requirement{Messaging}
\label{fr:messaging}

\begin{reqtable}
    Description        & 
                        All registered users should have access to a messaging
                        inbox, which will display all their existing messaging
                        threads. All message threads will be sorted by the time
                        of the last recieved/sent message of that particular thread.

                        Each thread should be unique to a listing item and will
                        be initiated whenever a user commits to collect an item.
                        \\
    \hline
    Priority           & High\\
    \hline
    Dependencies       & \autoref{fr:error-reporting},
    \autoref{fr:listing},
    \autoref{fr:commit}\\
    \hline
    Expected results   & Messaging facility operates as specified in the
                        description above.\\
    \hline
    Exception Handling & 
                        \begin{description}
                            \itemsep0em
                            \item [Messaging facility fails to meet specification criteria:]
                                Report error to webmaster.
                        \end{description}
                        \\
    \hline
\end{reqtable}


\requirement{Orders Page}
\label{fr:orders}

\begin{reqtable}
    Description        & 
                        All registered users should have access to an orders
                        page. This will contain:

                        \begin{description}
                            \itemsep0em
                            \item [Current orders:] Listings which the user has committed
                                to buying and currently have an “active” status.
                                A listing should be added to the user’s current orders
                                once he commits to collecting the item.
                            \item [Order history:] Listings which the user once committed
                                to buying, and now have an “inactive” status. The user
                                should be able to rate the publisher once the listing is
                                made inactive.
                        \end{description}

                        \\
    \hline
    Priority           & High\\
    \hline
    Dependencies       & \autoref{fr:error-reporting},
    \autoref{fr:listing}\\
    \hline
    Expected results   & User can access orders page.

                        Orders correctly categorised in "Current" and "History".
                        \\
    \hline
    Exception Handling & 
                        \begin{description}
                            \itemsep0em
                            \item [User has no orders:] Orders Page will be empty.
                        \end{description}
                        \\
    \hline
\end{reqtable}


\requirement{Watch-list}
\label{fr:watchlist}

\begin{reqtable}
    Description        & All users should have a watch-list, to which they can
                        add any item from the catalogue
                        which they are interested in but do not yet wish to
                        commit to receive.

                        Any item added to the watchlist should be removed if:

                        \begin{itemize}
                            \itemsep-1em
                            \item The user commits to the item.
                            \item The item becomes inactive
                        \end{itemize}
                        
                        The user should also have the option to manually remove
                        an item from the watchlist.
                        \\
    \hline
    Priority           & Low\\
    \hline
    Dependencies       & \autoref{fr:error-reporting}
                        \autoref{fr:item-catalogue}\\
    \hline
    Expected results   & Users can add and remove items from watch-list\\
    \hline
    Exception Handling & 
                        \begin{description}
                            \itemsep0em
                            \item [Watch-list not functioning as specified:]
                                Report errot to webmaster.
                        \end{description}
                        \\
    \hline
\end{reqtable}


\pagebreak
\subsection{Non-Functional Requirements}
%\begin{itemize}
    %\item Items will be stored in only one location and so will be consistent.
    %\item Animations and transitions will be simple and fluent in order to keep
        %the interface responsive.
    %\item All functionalities will be supported on all devices, apart from
        %wearables.
%\end{itemize}

\begin{table}[ht!]
    \centering
    \begin{tabular}{|p{2.5cm}|p{12cm}|}
        \hline
        Date & Task\\
        \hline
        \multirow{3}{2.5cm}{Usability} & The User will be able to easily access
        and use the website without need for prior knowledge, due to its simple
        and intuitive design. A Survey to Users will be distributed to get user
        opinion on how to increase usability.\\
        \cline{2-2}
        & A help page will also be available to assist the User with any questions they may have. \\
        \cline{2-2}
        & The Site should be useable on a variety of devices including desktop
        and mobile devices. As well as a variety of different commonly used
        browsers such as Google Chrome, Safari and Internet Explorer.\\
        \hline
        \multirow{4}{2.5cm}{Reliability} & Every link within the website goes to
        its correct location on every use and works correctly.\\
        \cline{2-2}
        & The system is up and running 24 hours of the day, with pre-warning if
        a system update is to be scheduled.\\
        \cline{2-2}
        & The Server would be up and running 90\% of the time, allowing for
        server updates.\\
        \cline{2-2}
        & The Server should be backed up regularly in case of a server crash
        where data can be lost.\\
        \hline
        \multirow{4}{2.5cm}{Security} & The User Passwords are stored hashed\\
        & A Password Limit increases the security for the user. A minimum of 8
        characters are required.\\
        \cline{2-2}
        & User Information will not be shared with any other company and no
        classified information such as address and contact details should be
        available for other users to see.\\
        \cline{2-2}
        & An age limit of over 18 is required, especially due to the anonymous
        nature of the website.\\
        \hline
        \multirow{2}{2.5cm}{Perfmormance} & Each page load within 5 seconds.\\
        & Data Updates to improve site quality to take place monthly.\\
        \hline
        \multirow{2}{2.5cm}{Style} & The colour scheme is simple and the font
        easy to read. Making the site accessible to the visually impaired as
        well as easy to use by any user. Unique Logo that makes the site easily
        recognisable to anyone who enters the site.\\
        \cline{2-2}
        & Sign up forms are separated across multiple pages. Making sign up
        easier due to cut down of text.\\
        \hline
        Extensibility & For Future versions of the Site there is potential to
        add a bar-code function to allow easy upload of products.\\
        \hline

    \end{tabular}
    \label{tab:nfr}
\end{table}

\section{Development Approach}

\subsection{Development Strategy}

We plan to execute the project using the agile method of development.

We first design the entire project, which includes wireframes, layouts, color schemes, database structure, soft and hard deadlines, milestones etc. We then implement our plan on the frontend and backend simultaneously, with each task being completed as sprints and tested as it is completed. The frontend and the backend are then combined and tested as a whole, which has been elaborated later under Testing Strategy.

\subsection{Development Stack}


This lists all kinds of tools, programming languages, frameworks, technologies etc. that will be used in the development of the project:


\begin{itemize}
    
    \item \textbf{PHP} - Database and other Backend Features

    \item \textbf{MySQL} - Data Storage
    
    \item \textbf{HTML/CSS}- Frontend
    
    \item \textbf{Material Bootstrap} - Responsive and attractive design using Google's Material guidelines
    
    \item \textbf{Javascript/JQuery} - Dynamic Client-Side Features
    
    \item \textbf{AJAX} - Communication between components
    
    \item \textbf{Integrated Development Environment (IDE)} - PHPStorm

\end{itemize}

\subsection{Testing Strategy}

The individual components are tested by the developer as soon as they are developed, along with some external testing by another team member. Once all components are integrated and the project is ready, it is tested rigorously as a whole for all possible combinations of inputs, security breach attempts, display issues, compatibility, faults and loopholes.

\subsection{Collaboration}

This lists all media used by the team members for collaboration and discussion:

\begin{itemize}

\item \textbf{Github} - Version Control 


\item \textbf{Slack} - Messaging app and website for teams


\item \textbf{Email} - Mailing list provided by university


\item \textbf{Group Practical} - The weekly group practical gives times for short meetings.

\end{itemize}

\section{Summary}

Our system will be reliable, attractive and above all easy to use. This will ensure that, instead of simply throwing away excess food, people will be more inclined to use our system with little extra effort and reduce food waste, while at the same time helping others. 

\begin{thebibliography}{1}

  \bibitem{wrap} WRAP UP  \href{http://www.wrap.org.uk/sites/files/wrap/Household_food_waste_in_the_UK_2015_Report.pdf}{http://www.wrap.org.uk}

  \bibitem{trussel} Trussell Trust \href{https://www.trusselltrust.org/2015/11/18/uk-foodbank-use-still-at-record-levels-as-hunger-remains-major-concern-for-low-income-families/}{https://www.trusselltrust.org}

  \bibitem{voth} Voices Of The Hungry \href{http://www.fao.org/in-action/voices-of-the-hungry/en/}{http://www.fao.org/in-action/voices-of-the-hungry/en/}
  
  \bibitem{USDA} USDA \href{https://www.fsis.usda.gov/wps/portal/fsis/topics/food-safety-education/get-answers/food-safety-fact-sheets/food-labeling/food-product-dating/food-product-dating/!ut/p/a1/jVJdT8IwFP01PJYWhzB8W5YYQBkSooy9mLvtbmvStUtbmPjrLaCJGqa0T_fccz_OaWlCY5pI2PMSLFcSxDFORq9sxUaDScjmy8ngns2il9XyIQyZv751hO0fhMi7sr7jBOy_-vkVA270IlyUNGnAVoTLQtG4REtAmha1oXGhVE4MFGgPpIDMElMh2q-EgBQFl-Vn2GiV7xwndxZ1gRua_FyLDdydRd56OJ1HHlsOfxMu-HYmdBvjlJdCpadH2gYy9XwnUWOBGnV_px1cWdvc9ViPtW3brzHnGYgcuDj0M1X32HfZ-NagtkTjHkEY0lZgjxjXp49w1IXGZUGIA6kRJIG94jk5tfAm4_HQv7RFpYyl8aXptKmf4_fHYMr4U73xTfAB8mN6gw!!/#3}{usda.gov}
  
  \bibitem{govuk} DEFR \href{https://www.gov.uk/government/uploads/system/uploads/attachment_data/file/69316/pb132629-food-date-labelling-110915.pdf}{gov.uk} (page 8)
  
  \bibitem{ebay} eBay \href{http://ebay.co.uk}{http://ebay.co.uk}

  \bibitem{foodsharing} Foodsharing \href{https://foodsharing.de/}{https://foodsharing.de/}

  \bibitem{olio} Olio \href{https://olioex.com/}{https://olioex.com/}
  
  \bibitem{fareshare} FareShare \href{http://www.fareshare.org.uk/}{http://www.fareshare.org.uk/}
\end{thebibliography}
\end{document}

